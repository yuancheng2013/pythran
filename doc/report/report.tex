\documentclass[a4paper, 11pt]{article}

% French encoding
\usepackage[utf8]{inputenc}
\usepackage[frenchb]{babel}
\usepackage[autolanguage]{numprint}
\usepackage[T1]{fontenc}
\usepackage{todonotes}

% basic pkgs
\usepackage{latexsym, amsmath, amssymb}
\usepackage{url}
\usepackage[colorlinks]{hyperref}

\newcommand\Pythran{\textsc{Pythran}}
\newcommand\Python{\textsc{Python}}
\newcommand\Pypy{\textsc{Pypy}}
\newcommand\Github{\textsc{Github}}

\begin{document}

% Cover
\title{Rapport d'étonnement}
\author{PENG Yuancheng}
\maketitle

% Contents
\hypersetup{linkcolor=black}
\tableofcontents
\setcounter{tocdepth}{3}

\bibliographystyle{plain} %Choose a bibliograhpic style

\section{Objectifs du stage}

Ce de stage de formation a pour objet de permettre à l'étudiant de mettre en
pratique les outils théoriques et méthodologiques acquis au cours de sa
formation, d'identifier ses compétences et de conforter son objectif
professionnel.

Le stage a ainsi pour but de préparer l'étudiant à l'entrée dans la vie active
par une meilleure connaissance de l'organisme d'accueil.

Le stage s'inscrit dans le cadre de la formation et du projet personnel et
professionnel de l'étudiant. Il entre dans son cursus pédagogique.

Le programme du stage est établi par Télécom Bretagne en fonction du programme
général de la formation dispensée. \footnote{Article 2 de la Convention de stage}

\section{Déroulement du stage}
\label{sec:deroulement-stage}

Le stage a commencé depuis le 1 juillet et a duré pendant deux mois.
Comme mon tuteur Serge Guelton est un chercheur associé qui ne travaille
pas tous le temps dans l'école, j'ai passé plus de la moitié de temps en
télétravail. Le reste du temps, j'ai travaillé dans son bureau ou à son domicile.

\subsection*{Sujet et missions du stage}

Pour bien connaitre mon sujet du stage, il faut tout d'abord savoir ce qu'est
\Pythran{} \cite{pythran}. \Pythran{} est un compilateur qui traduit un sous-ensemble du
langage \Python{} en \emph{C++} avec pour objectif d'obtenir des codes qui
s'exécutent plus rapidement.

Mon sujet du stage est l'expérimentation et la validation de performances
autour du compilateur \Pythran{}. Les objectifs sont~:

  \begin{itemize}
    \item Constituer une base de \emph{benchmarks} adaptée.
    \item Évaluer le logiciel \Pythran{} sur ces \emph{benchmarks} dans le temps.
    \item Comparer les performances de \Pythran{} à d'autres outils similaires.
  \end{itemize}

	Ce rapport va présenter une divers aspects du monde \emph{Open Source}
	qui m'a renouvelé le point de vue de la vie active d'un ingénieur
	informatique et va aussi présenter les résultats de mon stage.

\section{Développement \emph{Open Source}}
\label{sec:developpement-open-source}

\subsection*{Développement distribué}

Comme \Pythran{} est un logiciel \emph{Open Source}, pour que tous le monde
puisse avoir l'opportunité de participer au développement ou de savoir les
techniques qu'il a appliquées, les sources du projet sont déposées sur
\Github{} auquel tout le monde peut accéder et partager les développements.

La source du projet est contrôlée par Git \cite{git}, un outil de gestion de version
décentralisé qui nous permet de travailler de manière distribué.


Grâce à \Github{}, on peut chacun développer sa partie du travail à part et à
la fin mettre les différents parties en commun s'ils sont acceptées par le
projet, c'est à dire après revue de code. C'est pour cette raison que le lieux et le temps de travail ne sont plus aussi
importants.

\subsubsection*{Validation du code}

Dans un grand projet \emph{Open Source}, on ne peut pas s'assurer que les \emph{pull
requests} demandés par les contributeurs soient corrects tout le temps. Il nous
faut donc un mécanisme dans le projet pour garantir que les changements du code sont
valides.

Dans le le projet \Pythran{}, chaque fois  qu'une nouvelle fonctionnalité est ajoutée ou
qu'un \emph{bug} est corrigé, il faut que ces changements passent la validation,
c'est-à-dire tous les cas tests qui existent déjà dans le projet restent valides.


\subsection*{Communication entre développeurs}

Le compilateur \Pythran{} est développé en Anglais, ce qui inclut code,
commentaires, et les discussions sur \Github{}.
Les développeurs discutent quotidiennement sur IRC (\emph{Internet Relay Chat}), et utilisent une liste de diffusion spécifique.

\section{Résultats}
\label{sec:resultats}

Pour évaluer et améliorer la performance de \Pythran{}, on a développé
divers outils et en a évalué différents aspects.

\subsection*{Pythran-replay}

Pythran-replay \cite{pythran-replay} est l'un des principaux résultats de mon stage~:
j'y ai investit environ la moitié de mon temps. C'est un script
\Python{} qui évalue les performances des \emph{commits} dans l'historique de \Pythran{}.
Il nous aide à trouver les \emph{commits} qui ont améliorés les
performances du logiciel et les \emph{commits} qui les ont dégradées.


En développant ce script intéressant, j'ai pu découvrir l'utilisation de
\Github{} et les principes du développement d'un projet en contrôlant ses versions.

Je remercie particulièrement Pierrick Brunet qui m'a donné énormément de
commentaires sur \emph{Github} et m'a grandement aidé à améliorer le code de Pythran-replay et
trouver des \emph{bugs} perturbants.

\subsection*{Pythran-benchmarks}

Pythran-benchmarks est aussi un script, mais il évalue \Pythran{} d'une autre
façon~:
il compare le temps d'exécution entre \Pythran{}, \Python{} et \Pypy{},
un des concurrents de \Pythran{}. Ce script s'appuie sur le \emph{benchmark}
\emph{euler} qui existe déjà dans le projet \Pythran{}.
Le benchmark s'agit d'un ensemble de problème de calculs informatique et mathématique
qui demande plus que méthode mathématique à résoudre.  Il a permis de trouver plusieurs
cas tests où \Pythran{} ne fonctionne pas bien.
Ces \emph{bugs} ont tous été corrigés par la communauté depuis.

\subsection*{Contribution au monde \emph{Open Source}}
Python-benchmarks \cite{python-benchmarks} est un projet qui compare les performances sur une divers de
benchmarks parmi les différents compilateur de python où son repo se trouve 
sur \emph{Github}.

Vue que l'évaluation de pythran ainsi que comparer avec des autres compilateurs
de python est l'un des mes objectifs de stage. J'ai donc contribué des
benchmarks à ce projet.


\subsection*{Compétences acquises}

J'ai grandement bénéficié de ce stage. Il m'a non seulement fait
apprendre beaucoup de connaissance en programmation, mais m'a aussi ouvert la
porte du monde de l'\emph{open source}. J'ai touché beaucoup d'outils très utiles sous Linux et de
techniques avancés.

Voici les outils que j'ai le plus apprécié pendant ce stage~:

\begin{description}
  \item[Linux] \hfill
    Un système d'exploitation excellent.
  \item[Vim] \hfill
    Un vrai éditeur pour programmeurs.
  \item[Git] \hfill
    Un outil qui nous aide à constituer grands projets.
  \item[\Python{}] \hfill
    Un des langage le plus efficace pour ingénieurs.
  \item[\LaTeX] \hfill
    Un outil qui produit des jolies documents en pdf.
\end{description}

En faisant ce stage, j'ai compris que la capacité de collaboration 
est aussi importante que les compétences techniques. En plus,pour que les
ingénieurs puissent communiquer entre eux sans difficulté, il est essentiel pour
nous de respecter les règles de programmation.

En plus, je remercie monsieur Thierry Chonavel qui m'a financé et a assisté ma
présentation de fin du stage, ainsi que mon tuteur Serge Guelton qui m'a orienté
pendant plus que 2 mois et m'a donne de formation pédagogique, puis tous les
gens sur le \emph{channel} pythran de la liste Freenode qui m'a aidé régler les
problèmes confondants.

Hors cela, j'ai aussi progressé mes niveaux des langues. Mon anglais a bien
progressé au niveaux d'écrire de la documentation techniques, et je peut parler
un peu plus couramment le français sur des sujets informatiques.

\section{Conclusion}
\label{sec:conclusion}
% Connaissance théorique et Problèmes pratiques
En développant des sources j'ai apprit comment coopérer avec des autres
informaticiens dans un grand projet compliqué. J'ai aussi constaté qu'il 
y a un grand écart entre les connaissances théoriques et les problèmes pratiques.

Pour d'être un programmeurs excellent, il faut connaitre encore plus de
connaissances sur une divers de domaines informatiques. Et ça nous demande de
la compétence de pouvoir apprendre rapidement des nouvelles technologies.

Mon encadrant est vraiment très gentil, il a relu avec étonnement mon rapport d'étonnement

\bibliography{Master}
\end{document}


