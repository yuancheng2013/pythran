\documentclass[a4paper, 11pt]{article}

% French encoding
\usepackage[utf8]{inputenc}
\usepackage[frenchb]{babel}
\usepackage[autolanguage]{numprint}
\usepackage[T1]{fontenc}

% basic pkgs
\usepackage{latexsym, amsmath, amssymb}
%\usepackage{indentfirst}

% settings
%\setlength{\parskip}{1em}
%\setlength{\parindent}{0pt}

\newcommand\Pythran{\textsc{Pythran}}
\newcommand\Python{\textsc{Python}}
\newcommand\Pypy{\textsc{Pypy}}
\newcommand\Github{\textsc{Github}}

\begin{document}

% Cover
\title{Rapport d'étonnement}
\author{PENG Yuancheng}
\maketitle

% Contents
\tableofcontents
\setcounter{tocdepth}{3}
\pagebreak

\section{Déroulement du stage}
\label{sec:deroulement-stage}

Le stage a commencé depuis le 1 juillet et a duré pendant deux mois.
Comme mon tuteur monsieur Serge Guelton est un chercheur associé qui ne travaille
pas tous le temps dans l'école, on a passé plus qu'un moitié de temps à
télétravail. Le reste de temps le lieu de travail n'était pas fixe non plus:
je travaillais soit dans le bureau soit dans sa maison.

\subsection*{Sujet et objectifs du stage}

Pour bien connaitre mon sujet du stage, il faut tout d'abord savoir qu'est-ce
que \Pythran{}. \Pythran{} est un compilateur qui traduit un sous-ensemble du
langage \Python{} en \emph{C++} pour le but que les code \Python{} exécutent plus
efficacement.

Mon sujet du stage est l'expérimentation et la validation de performances
autour du compilateur \Pythran{}, où les objectifs sont:

  \begin{itemize}
    \item Constituer une base de benchmarks adaptée
    \item Évaluer le logiciel \Pythran{} sur ces benchmarks dans le temps
    \item Comparer les performances de \Pythran{} à d'autres outils similaires
  \end{itemize}

\section{Développement \emph{Open Source}}
\label{sec:developpement-open-source}

\subsection*{Développement distribué}

Comme \Pythran{} est un logiciel \emph{Open Source}, pour que tous le monde puisse
avoir l'opportunité à participer le développement ou à savoir les techniques
qu'il a appliqué, la source est déposé sur \Github{} où tout le monde peut
y accéder et voir son développement.

La source du projet est contrôlée par Git, un outil de gestion de version
décentralisé qui nous permet de travailler de manière distribué.

\subsection*{Communication entre développeurs}

Le compilateur \Pythran{} est développé tous en Anglais, incluant son code,
commentaires, et les discutions sur \Github{}.
Les développeurs discutent quotidiennement sur IRC et plusieurs mailing liste.
%TODO référence, bibliothèques

\section{Résultats}
\label{sec:resultats}

Pour évaluer et améliorer la performance de \Pythran{}, on a développé
divers outils et en a évalué aux différents aspects.

\subsection*{Pythran-replay}

Pythran-replay un des principaux travails pendant mon stage. Il est un script
\Python{} qui évalue les performances des commits historiques du \Pythran{}.
Il nous aide à trouver à quels commits qu'on a amélioré la
performance du logiciel et les commits où on en a perdu.

En développant ce script intéressant, j'ai bien découvert l'utilisation de
\Github{} et le principe du développement d'un projet en contrôlant ses versions.

\subsection*{Pythran-benchmarks}

Pythran-benchmarks est aussi un script, mais il évalue \Pythran{} d'un autre
façon: il compare le temps d'exécution entre \Pythran{} \Python{} et \Pypy{},
un des concurrents de \Pythran{}. Ce script a appuye sur le benchmarck
\emph{euler} qui existe deja dans le projet \Pythran{}. Il a trouve plusieurs
test cases où \Pythran{} n'a pas bien fonctionné.

\subsection*{Contribution à \emph{Open Source}}
%TODO


\subsection*{Compétences acquises}
Je suis grandement bénéficié de ce stage. Il non seulement m'a fait
apprendre beaucoup de connaissance de programmation, mais aussi m'a ouvert la
porte open source.

Voici des outils j'apprécie le plus pendant ce stage:

\begin{description}
  \item[Linux] \hfill
    Un système d'exploitation excellent.
  \item[Vim] \hfill
    Un vrai éditeur pour programmeurs.
  \item[Git] \hfill
    Un outil qui nous aide à constituer grands projets.
  \item[\Python{}] \hfill
    Un des langage le plus efficace pour ingénieurs.
  \item[\LaTeX] \hfill

\end{description}

% TODO Anglais et Français


\newpage
\section{Conclusion}
\label{section4}
% Connaissance théorique et Problèmes pratiques
% Choix futurs

\end{document}
