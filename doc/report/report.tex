\documentclass[a4paper, 11pt]{article}

% French encoding
\usepackage[utf8]{inputenc}
\usepackage[frenchb]{babel}
\usepackage[autolanguage]{numprint}
\usepackage[T1]{fontenc}

% basic pkgs
\usepackage{latexsym, amsmath, amssymb}
\usepackage{url}

\newcommand\Pythran{\textsc{Pythran}}
\newcommand\Python{\textsc{Python}}
\newcommand\Pypy{\textsc{Pypy}}
\newcommand\Github{\textsc{Github}}

\begin{document}

% Cover
\title{Rapport d'étonnement}
\author{PENG Yuancheng}
\maketitle

% Contents
\tableofcontents
\setcounter{tocdepth}{3}
\pagebreak

\section{Objectifs du stage}

Ce de stage de formation a pour objet de permettre à l'étudiant de mettre en
pratique les outils théoriques et méthodologiques acquis au cours de sa
formation, d'identifier ses compétences et de conforter son objectif
professionnel.

Le stage a ainsi pour but de préparer l'étudiant à l'entrée dans la vie active
par une meilleure connaissance de l'organisme d'accueil.

Le stage s'inscrit dans le cadre de la formation et du projet personnel et
professionnel de l'étudiant. Il entre dans son cursus pédagogique.

Le programme du stage est établi par Télécom Bretagne en fonction du programme
général de la formation dispensée. \footnote{Article 2 de la Convention de stage}

\section{Déroulement du stage}
\label{sec:deroulement-stage}

Le stage a commencé depuis le 1 juillet et a duré pendant deux mois.
Comme mon tuteur monsieur Serge Guelton est un chercheur associé qui ne travaille
pas tous le temps dans l'école, on a passé plus qu'un moitié de temps à
télétravail. Le reste de temps les lieux de travail étaient aussi flexibles:
je travaillais soit dans le bureau soit dans sa maison.

\subsection*{Sujet et missions du stage}

Pour bien connaitre mon sujet du stage, il faut tout d'abord savoir qu'est-ce
que \Pythran{} \footnote{La source de pythran se trouve sur
https://github.com/serge-sans-paille/pythran}.
\Pythran{} est un compilateur qui traduit un sous-ensemble du
langage \Python{} en \emph{C++} pour le but que les code s'écrit en \Python{}
exécutent plus efficacement.

Mon sujet du stage est l'expérimentation et la validation de performances
autour du compilateur \Pythran{}, où les missions sont:

  \begin{itemize}
    \item Constituer une base de benchmarks adaptée
    \item Évaluer le logiciel \Pythran{} sur ces benchmarks dans le temps
    \item Comparer les performances de \Pythran{} à d'autres outils similaires
  \end{itemize}


\section{Développement \emph{Open Source}}
\label{sec:developpement-open-source}

\subsection*{Développement distribué}

Comme \Pythran{} est un logiciel \emph{Open Source}, pour que tous le monde puisse
avoir l'opportunité à participer le développement ou à savoir les techniques
qu'il a appliqué, la source du projet est déposé sur \Github{} où tout le monde peut
y accéder et voire observer son développement.

La source du projet est contrôlée par Git, un outil de gestion de version
décentralisé qui nous permet de travailler de manière distribué.

Grâce à \Github{}, on peut chacun développer son partie de travail à part et à
la fin mettre les différents parties tout ensemble s'ils sont acceptés par le
projet. C'est pour cette raison que le lieux et le temps de travail ne sont plus
importants.

\subsubsection*{Validation de code}
Dans un grand projet \emph{Open Source}, on ne peut pas s'assurer que les pull
requests demandés par les contributeurs soient corrects tout le temps. Il nous
faut donc un mécanisme dans le projet à garantir les changements du code sont
valides.

Dans le le projet \Pythran{}, chaque fois après avoir ajouté une nouvelle fonction ou
une fixé une bug, il faudrait que ces changements passent la validation,
c'est-à-dire tous les test cases qui existent déjà dans le projet.


\subsection*{Communication entre développeurs}

Le compilateur \Pythran{} est développé tous en Anglais, incluant son code,
commentaires, et les discutions sur \Github{}.
Les développeurs discutent quotidiennement plusieurs mailing liste et aussi sur 
\emph{IRC} \footnote{introduction d'IRC et de la liste Freenode sont disponible
sur Wikipédia : http://en.wikipedia.org/wiki/Internet\_Relay\_Chat}, dont le 
channel est le pythran de la liste \emph{Freenode}.

\section{Résultats}
\label{sec:resultats}

Pour évaluer et améliorer la performance de \Pythran{}, on a développé
divers outils et en a évalué aux différents aspects.

\subsection*{Pythran-replay}

Pythran-replay\footnote{https://github.com/yuancheng2013/pythran-replay}
l'un des principaux travails pendant mon stage dont j'ai
contribue presque un moitié de temps dessus. Il est un script
\Python{} qui évalue les performances des commits historiques du \Pythran{}.
Il nous aide à trouver les commits qu'on a amélioré la
performance du logiciel et les commits où on en a perdu.

En développant ce script intéressant, j'ai bien découvert l'utilisation de
\Github{} et le principe du développement d'un projet en contrôlant ses versions.

Je remercie particulièrement Pierrick Brunet qui m'a donné énormément de
commentaires sur \emph{Github} et m'a grandement aidé améliorer le code de Pythran-replay et
trouver des bugs perturbants.

\subsection*{Pythran-benchmarks}

Pythran-benchmarks est aussi un script, mais il évalue \Pythran{} d'un autre
façon: il compare le temps d'exécution entre \Pythran{} \Python{} et \Pypy{},
un des concurrents de \Pythran{}. Ce script a appuyé sur le benchmark
\emph{euler} qui existe déjà dans le projet \Pythran{}. Il a trouve plusieurs
test cases où \Pythran{} n'a pas bien fonctionné.

\subsection*{Contribution à \emph{Open Source}}
Python-benchmarks \footnote{Pour plus d'information: https://github.com/numfocus/python-benchmarks}
est un projet qui compare les performances sur une divers de
benchmarks parmi les différents compilateur de python où son repo se trouve 
sur \emph{Github}.

Vue que l'évaluation de pythran ainsi que comparer avec des autres compilateurs
de python est l'un des mes objectifs de stage. J'ai donc contribué des
benchmarks à ce projet.


\subsection*{Compétences acquises}
Je suis grandement bénéficié de ce stage. Il non seulement m'a fait
apprendre beaucoup de connaissance de programmation, mais aussi m'a ouvert la
porte open source. J'ai touché beaucoup d'outils très utiles sous Linux et de
techniques avancés.

Voici des outils que j'apprécie le plus pendant ce stage:

\begin{description}
  \item[Linux] \hfill
    Un système d'exploitation excellent.
  \item[Vim] \hfill
    Un vrai éditeur pour programmeurs.
  \item[Git] \hfill
    Un outil qui nous aide à constituer grands projets.
  \item[\Python{}] \hfill
    Un des langage le plus efficace pour ingénieurs.
  \item[\LaTeX] \hfill
    Un outil qui produit des jolies documents en pdf.
\end{description}


Hors cela, j'ai aussi progresse mon niveaux des langues. Mon anglais a bien
progressé au niveaux d'écrire de la documentation techniques, et je peut parler
un peu plus couramment le français sur des sujets informatiques.

\section{Conclusion}
\label{sec:conclusion}
% Connaissance théorique et Problèmes pratiques
En développant des sources j'ai apprit comment coopérer avec des autres
informaticiens dans un grand projet compliqué. J'ai aussi constaté qu'il 
y a un grand écart entre les connaissances théoriques et les problèmes pratiques.

Pour d'être un programmeurs excellent, il faut connaitre encore plus de
connaissances sur une divers de domaines informatiques. Et ça nous demande de
la compétence de pouvoir apprendre rapidement des nouvelles technologies.

\end{document}


